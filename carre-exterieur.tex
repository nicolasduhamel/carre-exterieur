\documentclass{amsart}

\usepackage[utf8]{inputenc}
\usepackage[T1]{fontenc}
\usepackage[francais]{babel}
\usepackage{bbm}
\usepackage{amssymb}
\usepackage{amsmath}
\usepackage{amsthm}
\usepackage{mathtools}
\usepackage{hyperref}
\usepackage{graphics}
\usepackage{enumerate}

\usepackage{eulervm}

\newtheorem{proposition}{Proposition}
\newtheorem{propriete}{Propriété}
\newtheorem{definition}{Définition}
\newtheorem{theoreme}{Théorème}
\newtheorem{lemme}{Lemme}

\DeclareMathOperator{\Hom}{\mathnormal{Hom}}
\DeclareMathOperator{\Ext}{\mathnormal{Ext}}

\begin{document}

\title{Facteurs $\gamma$ du carré extérieur}
\date{\today}
\maketitle

\section{Préliminaires}
 
 \begin{proposition}
 \label{funcloc}
 Equation fonctionnelle locale
 \end{proposition}
 
 \begin{proposition}
 \label{funcglob}
 Equation fonctionnelle globale
 \end{proposition}
 
 \begin{proposition}
 \label{globalisation}
 Globalisation
 \end{proposition}
 
 \section{Facteurs $\gamma$}
 
 On se propose de démontrer l'égalité entre les facteurs $\gamma^{JS}(., \pi, \Lambda^2, \psi)$ et $\gamma^{Sh}(., \pi, \Lambda^2, \psi)$ à une constante (dépendant de $\pi$) de module 1 près.
 
 On commence à montrer cette égalité pour les facteurs $\gamma$ archimédiens. Pour le moment, les résultats connus ne nous donnent même pas l'existence du facteur $\gamma^{JS}$ dans le cas archimédien, ce sera une conséquence de la méthode de globalisation.
 
 \begin{proposition}
 Soit $F = \mathbb{R}$ ou $\mathbb{C}$. Soit $\pi$ une représentation tempérée irréductible de $GL_{2n}(F)$. 
 
 Il existe une fonction méromorphe $\gamma^{JS}(s,\pi,\Lambda^2,\psi)$ telle que pour tous $s$, $W \in W(\pi, \psi)$ et $\phi \in \mathcal{S}(F)$, on ait
 \begin{equation}
 \gamma^{JS}(s, \pi, \Lambda^2, \phi) J(s, W, \phi) = J(1-s, \rho(w_{n,n})\tilde{W}, \mathcal{F}_\psi(\phi)).
 \end{equation}
 
 De plus, il existe une constante $c(\pi)$ de module 1 telle que pour tout $s \in \mathbb{C}$,
 \begin{equation}
 \gamma^{JS}(s, \pi, \Lambda^2, \psi) = c(\pi)\gamma^{Sh}(s, \pi, \Lambda^2, \psi).
 \end{equation}
 \end{proposition}
 
 \begin{proof}
 Soit $k$ un corps de nombres, on suppose que $k$ a une seule place archimédienne, elle est réelle (respectivement complexe) lorsque $F=\mathbb{R}$ (respectivement $F=\mathbb{C}$); par exemple, $k=\mathbb{Q}$ si $F=\mathbb{R}$ et $k=\mathbb{Q}(i)$ si $F=\mathbb{C}$. Soit $v \neq v'$ deux places non archimédiennes distinctes, soit $U \subset Temp(GL_{2n}(F))$ un ouvert contenant $\pi$.
 
 D'après la proposition \ref{globalisation}, il existe une représentation automorphe cuspidale $\Pi$ telle que $\Pi_{\infty} \in U$ et $\Pi_w$ soit non ramifiée pour toute place non archimédienne $w \neq v$.
 
 D'après la proposition \ref{funcglob}, on a
 \begin{equation}
 \begin{split}
 &J(s, W_\infty, \phi_\infty)J(s, W_v, \phi_v)L^S(s, \Pi, \Lambda^2) \\
 &= J(1-s, \rho(w_{n,n})\tilde{W}_\infty, \mathcal{F}_\psi(\phi_\infty))J(1-s, \rho(w_{n,n})\tilde{W}_v, \mathcal{F}_\psi(\phi_v))L^S(1-s, \tilde{\Pi}, \Lambda^2)
 \end{split}
 \end{equation}
 et
 \begin{equation}
 L^S(s, \Pi, \Lambda^2) = \gamma^{Sh}(s, \Pi_\infty, \Lambda^2, \psi_\infty)\gamma^{Sh}(s, \Pi_v, \Lambda^2, \psi_v)L^S(1-s, \tilde{\Pi}, \Lambda^2).
 \end{equation}
 
 Le quotient de ces deux équations nous donne, en utilisant la proposition \ref{funcloc} sur $\Pi_v$, la relation
 \begin{equation}
 \frac{J(1-s, \rho(w_{n,n})\tilde{W}_\infty, \mathcal{F}_\psi(\phi_\infty))}{J(s, W_\infty, \phi_\infty)\gamma^{Sh}(s, \Pi_\infty, \Lambda^2, \psi_\infty)} \frac{\gamma^{JS}(s, \Pi_v, \Lambda^2, \psi_v)}{\gamma^{Sh}(s, \Pi_v, \Lambda^2, \psi_v)} = 1.
 \end{equation}
 
 Ce qui prouve la première partie de la proposition pour $\Pi_\infty$, l'existence du facteur $\gamma^{JS}(s, \Pi_\infty, \Lambda^2, \psi_\infty)$.
 
 On choisit maintenant pour $U$ une base de voisinage contenant $\pi$, en utilisant la continuité des facteurs $\gamma$ et des facteurs $J$, on en déduit que $\frac{J(1-s, \rho(w_{n,n})\tilde{W}, \mathcal{F}_\psi(\phi))}{J(s, W, \phi)}$
 est une fonction méromorphe indépendante de $W$ et de $\phi$, que l'on note $\gamma^{JS}(s, \pi, \Lambda^2, \psi)$, qui est le produit de $\gamma^{Sh}(s, \pi, \Lambda^2, \psi)$ et d'une fonction, que l'on note $R(s)$, qui est limite de fractions rationnelles en $q_v^s$; donc $R$ est une fonction périodique de période $\frac{2i\pi}{\log q_v}$.
 
  On réutilisant notre raisonnement en la place $v'$, on voit que $R$ est aussi périodique de période $\frac{2i\pi}{\log q_{v'}}$; donc est constante. Ce qui nous permet de voir qu'il existe une constante $c(\pi)=R$ telle que
 \begin{equation}
 \gamma^{JS}(s, \pi, \Lambda^2, \psi) = c(\pi)\gamma^{Sh}(s, \pi, \Lambda^2, \psi).
 \end{equation}
 
 Il ne nous reste plus qu'à montrer que la constante $c(\pi)$ est de module 1. Reprenons l'équation fonctionnelle locale archimédienne,
 \begin{equation}
 \label{funcarch}
 \gamma^{JS}(s, \pi, \Lambda^2, \phi) J(s, W, \phi) = J(1-s, \rho(w_{n,n})\tilde{W}, \mathcal{F}_\psi(\phi)).
 \end{equation}
 
 On utilise maintenant l'équation fonctionnelle sur la représentation $\tilde{\pi}$ pour transformer le facteur $J(1-s, \rho(w_{n,n})\tilde{W}, \mathcal{F}_\psi(\phi))$, ce qui nous donne
 \begin{equation}
 \gamma^{JS}(s, \pi, \Lambda^2, \phi) J(s, W, \phi) = \frac{J(s, W, \mathcal{F}_{\bar{\psi}}(\mathcal{F}_\psi(\phi)))}{\gamma^{JS}(1-s, \tilde{\pi}, \Lambda^2, \bar{\phi})}.
 \end{equation}
 
 Puisque $\mathcal{F}_{\bar{\psi}}(\mathcal{F}_\psi(\phi)) = \phi$, on obtient donc la relation 
 \begin{equation}
 \gamma^{JS}(s, \pi, \Lambda^2, \phi)\gamma^{JS}(1-s, \tilde{\pi}, \Lambda^2, \bar{\phi}) = 1.
 \end{equation}
 
 D'autre part, en conjuguant l'équation \ref{funcarch}, on obtient
 \begin{equation}
 \overline{\gamma^{JS}(s, \pi, \Lambda^2, \phi)} = \gamma^{JS}(\bar{s}, \bar{\pi}, \Lambda^2, \bar{\phi}).
 \end{equation}
 
 Comme $\pi$ est tempérée, $\pi$ est unitaire, donc $\tilde{\pi} \simeq \bar{\pi}$. On en déduit, pour $s = \frac{1}{2}$,
 \begin{equation}
 |\gamma^{JS}(\frac{1}{2}, \pi, \Lambda^2, \psi)|^2=1.
 \end{equation}
 
 D'autre part, le facteur $\gamma$ de Shahidi vérifie aussi $|\gamma^{JS}(\frac{1}{2}, \pi, \Lambda^2, \phi)|^2=1$; on en déduit donc que $c(\pi)$ est bien de module 1.
 \end{proof}
\end{document}