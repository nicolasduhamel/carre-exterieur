\documentclass{amsart}

\usepackage[utf8]{inputenc}
\usepackage[T1]{fontenc}
\usepackage[francais]{babel}
\usepackage{bbm}
\usepackage{amssymb}
\usepackage{amsmath}
\usepackage{amsthm}
\usepackage{mathtools}
\usepackage{hyperref}
\usepackage{graphics}
\usepackage{enumerate}

\usepackage{eulervm}

\newtheorem{proposition}{Proposition}[section]
\newtheorem{propriete}{Propriété}[section]
\newtheorem{definition}{Définition}[section]
\newtheorem{theoreme}{Théorème}[section]
\newtheorem{lemme}{Lemme}[section]

\DeclareMathOperator{\Hom}{\mathnormal{Hom}}
\DeclareMathOperator{\Ext}{\mathnormal{Ext}}

\begin{document}

\title{Facteurs $\gamma$ du carré extérieur}
\date{\today}
\maketitle

Soit $F$ un corps local de caractéristique $0$, $\psi$ un caractère non trivial de $F$ et $\pi$ une représentation tempérée irréductible de $GL_{2n}(F)$. Jacquet et Shalika ont défini une fonction L du carré extérieur $L_{JS}(s, \pi, \Lambda^2)$ par des intégrales notées $J(s, W, \phi)$, où $W \in \mathcal{W}(\pi, \psi)$ est un élément du modèle de Whittaker de $\pi$ et $\phi \in \mathcal{S}(F^n)$ est une fonction de Schwartz. Matringe a prouvé que, lorsque $F$ est non archimédien, ces intégrales $J(s,W,\phi)$ vérifient une équation fonctionnelle, ce qui permet de définir des facteurs $\gamma$, que l'on note $\gamma^{JS}(s,\pi,\Lambda^2,\psi)$. 

On montre que l'on a encore une équation fonctionnelle lorsque $F$ est archimédien et que les facteurs $\gamma$ sont égaux à une constante de module 1 prés à ceux définis par Shahidi, que l'on note $\gamma^{Sh}(s,\pi,\Lambda^2,\psi)$. Plus exactement, il existe une constante $c(\pi)$ de module 1, telle que
\begin{equation}
\gamma^{JS}(s,\pi,\Lambda^2,\psi)=c(\pi)\gamma^{Sh}(s,\pi,\Lambda^2,\psi),
\end{equation}
pour tout $s \in \mathbb{C}$. La preuve se fait par une méthode de globalisation, on considère $\pi$ comme une composante locale d'une représentation automorphe cuspidale.

\section{Préliminaires}

\subsection{Théorie locale}
Les intégrales $J(s, W, \phi)$ sont définies par
 \begin{equation}
\int_{N_n\backslash{G_n}} \int_{Lie(B_n)\backslash{M_n}} W\left(\sigma \begin{pmatrix}
1 & X \\
0 & 1
\end{pmatrix}\begin{pmatrix}
g & 0 \\
0 & g
\end{pmatrix}\right)\psi(-Tr(X))dX\phi(e_ng)|\det g|^s dg
 \end{equation}
pour tous $W \in \mathcal{W}(\pi, \psi)$, $\phi \in \mathcal{S}(F^n)$ et $s \in \mathbb{C}$. On a noté $G_n$ le groupe $GL_n(F)$, $B_n$ le sous groupe des matrices triangulaires supérieures, $N_n$ le sous-groupe de $B_n$ des matrices dont les éléments diagonaux sont $1$ et $M_n$ l'ensemble des matrices de taille $n \times n$ à coefficients dans $F$. L'élément $\sigma$ est la matrice associée à la permutation $\bigl(\begin{smallmatrix}
    1 & 2 & \cdots & n & n+1 & n+2 & \cdots & 2n \\
    1 & 3 & \cdots &  2n-1  & 2 & 4 & \cdots & 2n
  \end{smallmatrix}\bigr).$
  
  Jacquet et Shalika ont démontré que ces intégrales convergent pour $Re(s)$ suffisamment grand, plus exactement, on dispose de la
  \begin{proposition}[Jacquet-Shalika]
  Il existe $\eta > 0$ tel que les intégrales $J(s, W, \phi)$ convergent absolument pour $Re(s) > 1 - \eta$.
  \end{proposition}
  
  Kewat montre, lorsque $F$ est p-adique, que ce sont des fractions rationnelles en $q^{s}$ où $q$ est le cardinal du corps résiduel de $F$. On aura aussi besoin d'avoir le prolongement méromorphe de ces intégrales lorsque $F$ est archimédien et d'un résultat de non annulation.
  \begin{proposition}[Belt]
  \label{nonzero}
  Fixons $s_0 \in \mathbb{C}$. Il existe $W \in \mathcal{W}(\pi, \psi)$ et $\phi \in \mathcal{S}(F^n)$ tels que $J(s,W,\phi)$ admet un prolongement méromorphe à tout le plan complexe et ne s'annule pas en $s_0$. Si $F=\mathbb{R}$ ou $\mathbb{C}$, le point $s_0$ peut éventuellement être un pôle. Si $F$ est $p$-adique, on peut choisir $W$ et $\phi$ tels que $J(s, W, \phi)$ soit entière.
  \end{proposition}
  
  Lorsque la représentation est non-ramifiée, on peut représenter la fonction L du carré extérieur obtenue par la correspondance de Langlands locale, que l'on note $L(s, \pi, \Lambda^2)$, (qui est égale à celle obtenue par la méthode de Langlands-Shahidi d'après un résultat d'Henniart \cite{henniart}) par ces intégrales.
  \begin{proposition}[Jacquet-Shalika]
  \label{calculnr}
  Supposons que $F$ est $p$-adique, le conducteur de $\psi$ est l'anneau des entiers $\mathcal{O}$ de $F$. Soit $\pi$ une représentation non ramifiée de $GL_{2n}(F)$. On note $\phi_0$ la fonction caractéristique de $\mathcal{O}^n$ et $W_0$ l'unique fonction de Whittaker invariante par $GL_{2n}(\mathcal{O})$ et qui vérifie $W(1)=1$. Alors
   \begin{equation}
   J(s,W_0,\phi_0) = L(s, \pi, \Lambda^2).
    \end{equation}
  \end{proposition}
  
  Pour finir cette section, on énonce l'équation fonctionnelle démontrée par Matringe lorsque $F$ est un corps $p$-adique. Plus précisément, on a la
 \begin{proposition}[Matringe]
 \label{funcloc}
 Supposons que $F$ est un corps $p$-adique et $\pi$ générique. Il existe un monôme $\epsilon(s,\pi,\Lambda^2,\psi)$ en $q^s$, tel que pour tous $W \in \mathcal{W}(\pi,\psi)$ et $\phi \in \mathcal{S}(F^n)$, ont ait
 \begin{equation}
 \epsilon(s, \pi, \Lambda^2, \psi) \frac{J(s,W,\phi)}{L(s,\pi,\Lambda^2)}  = \frac{J(1-s,\rho(w_{n,n})\tilde{W},\hat{\phi})}{L(1-s,\tilde{\pi},\Lambda^2)},
 \end{equation}
 où $\hat{\phi} = \mathcal{F}_\psi(\phi)$ est la transformée de Fourier de $\phi$ par rapport au caractère $\psi$ et $\tilde{W} \in \mathcal{W}(\tilde{\pi}, \bar{\psi})$ est la fonction de Whittaker définie par $\tilde{W}(g) = W(w_n(g^t)^{-1})$, avec $w_n$ la matrice associée à la permutation $\bigl(\begin{smallmatrix}
    1 & \cdots & 2n  \\
    2n & \cdots &  1 
  \end{smallmatrix}\bigr)$
  et
 $w_{n,n} = \begin{pmatrix}
0 & 1_n \\
1_n & 0
\end{pmatrix}$. On définit alors le facteur $\gamma$ de Jacquet-Shalika par la relation
\begin{equation}
\gamma^{JS}(s,\pi,\Lambda^2,\psi)  = \epsilon(s,\pi,\Lambda^2,\psi)\frac{L(1-s,\tilde{\pi},\Lambda^2)}{L(s,\pi,\Lambda^2)}.
\end{equation}
 \end{proposition}
 
  \subsection{Théorie globale}
  La méthode que l'on utilise est une méthode de globalisation. Essentiellement, on verra $\pi$ comme une composante locale d'une représentation automorphe cuspidale. Pour ce faire, on aura besoin de l'équivalent global des intégrales $J(s, W, \phi)$.
  
  Soit $K$ un corps de nombres et $\psi_\mathbb{A}$ un caractère non trivial de $\mathbb{A}_K/K$. Soit $\Pi$ une représentation automorphe cuspidale irréductible sur $GL_{2n}(\mathbb{A}_K)$. Pour $\varphi \in \Pi$, on considère
  \begin{equation}
  W_\varphi(g) = \int_{N_{2n}(K)\backslash{N_{2n}(\mathbb{A}_K)}} \varphi(ug)\psi_\mathbb{A}(u)du
  \end{equation}
  la fonction de Whittaker associée. On considère $\psi_\mathbb{A}$ comme un caractère de $N_{2n}(\mathbb{A}_K)$ en posant $\psi_\mathbb{A}(u) = \psi_\mathbb{A}(\sum_{i=1}^{2n-1} u_{i,i+1})$. Pour $\Phi \in \mathcal{S}(\mathbb{A}_K^n)$ une fonction de Schwartz, on note $J(s, W_\varphi, \Phi)$ l'intégrale
  \begin{equation}
\int_{N_n\backslash{G_n}} \int_{Lie(B_n)\backslash{M_n}} W_\varphi \left(\sigma \begin{pmatrix}
1 & X \\
0 & 1
\end{pmatrix}\begin{pmatrix}
g & 0 \\
0 & g
\end{pmatrix}\right)\psi_\mathbb{A}(Tr(X))dX\Phi(e_ng)|\det g|^s dg
 \end{equation}
 où l'on note $G_n$ le groupe $GL_n(\mathbb{A}_K)$, $B_n$ le sous groupe des matrices triangulaires supérieures, $N_n$ le sous-groupe de $B_n$ des matrices dont les éléments diagonaux sont $1$ et $M_n$ l'ensemble des matrices de taille $n \times n$ à coefficients dans $\mathbb{A}_K$.
 
  Finissons cette section par l'équation fonctionnelle globale démontrée par Jacquet et Shalika \cite{jacquet-shalika}.
 \begin{proposition}[Jacquet-Shalika]
 \label{funcglob}
 Les intégrales $J(s, W_\varphi, \Phi)$ convergent absolument pour $Re(s)$ suffisamment grand. De plus, $J(s, W_\varphi, \Phi)$ admet un prolongement méromorphe à tout le plan complexe et vérifie l'équation fonctionnelle suivante
 \begin{equation}
 J(s,W_\varphi,\Phi)=J(1-s, \rho(w_{n,n})\tilde{W}_\varphi, \hat{\Phi}),
 \end{equation}
 où $\tilde{W}_\varphi(g) = W_\varphi(w_n(g^t)^{-1})$ et $\hat{\Phi}$ est la transformée de Fourier de $\Phi$ par rapport au caractère $\psi_\mathbb{A}$.
 \end{proposition}
 
 Comme on peut s'y attendre, les intégrales globales sont reliées aux intégrales locales. Plus exactement, si $W=\prod_v W_v$ et $\Phi = \prod_v \Phi_v$, où $v$ décrit les places de $K$, on a
 \begin{equation}
 J(s,W_\varphi,\Phi)=\prod_v J(s, W_v, \Phi_v).
 \end{equation}
 
 \subsection{Globalisation}
 
 Comme la preuve se fait par globalisation, la première chose à faire est de trouver un corps de nombres dont $F$ est une localisation. On dispose du
 \begin{lemme}[Kable \cite{kable}]
 \label{corpsglobal}
 Supposons que $F$ est un corps $p$-adique. Il existe un corps de nombres $k$ et une place $v_0$ telle que $k_{v_0} = F$, où $v_0$ est l'unique place de $k$ au dessus de $p$.
 \end{lemme}
 
 On note $Temp(GL_{2n}(F))$ l'ensemble des classes d'isomorphismes de représentations tempérées irréductibles. On va définir une topologie sur $Temp(GL_{2n}(F))$. Soit $M$ un sous-groupe de Levi de $GL_{2n}(F)$ et $\sigma$ une représentation irréductible de carré intégrable de $M$, on note $X^*(M)$ le groupe des caractères algébriques de $M$, on dispose alors d'une application $\chi \otimes \lambda \in X^*(M) \otimes i\mathbb{R} \mapsto i^G_M(\sigma \otimes \chi_\lambda) \in Temp(GL_{2n}(F))$ où $\chi_\lambda(g) = |\chi(g)|^\lambda$. On définit alors une base de voisinage de $i^G_M(\sigma)$ dans $Temp(GL_{2n}(F))$ comme l'image d'une base de voisinage de $0$ dans $X^*(M) \otimes i\mathbb{R}$.
 
 Cette topologie sur $Temp(GL_{2n}(F))$ nous permet d'énoncer le résultat principal dont on aura besoin pour la méthode de globalisation.
 \begin{proposition}[Beuzart-Plessis]
 \label{globalisation}
 Soient $k$ un corps de nombres, $v_0,v_1$ deux places distinctes de $k$ avec $v_1$ non archimédienne. Soit $U$ un ouvert de $Temp(GL_{2n}(k_{v_0}))$. Alors il existe une représentation automorphe cuspidale irréductible $\Pi$ de $GL_{2n}(\mathbb{A}_k)$ telle que $\Pi_{v_0} \in U$ et $\Pi_v$ est non ramifiée pour toute place non archimédienne $v \not \in \{v_0,v_1\}$.
 \end{proposition}
 
 \subsection{Fonctions tempérées}
 On aura besoin dans la suite de connaître la dépendance que $J(s, W, \phi)$ lorsque l'on fait varier la représentation $\pi$. Pour ce faire, on introduit la notion de fonction tempérée et on étend la définition de $J(s,W,\phi)$ pour ces fonctions tempérées.
 
 L'espace des fonctions tempérées $C^w(N_{2n}(F)\backslash{GL_{2n}(F)}, \psi)$ est l'espace des fonctions $f : GL_{2n}(F) \rightarrow \mathbb{C}$ telles que $f(ng) = \psi(n)f(g)$ pour tous $n \in N_{2n}(F)$ et $g \in GL_{2n}(F)$, on impose les conditions suivantes :
 \begin{itemize}
 \item Si $F$ est $p$-adique, $f$ est localement constante et il existe $d > 0$ et $C > 0$ tels que $|f(nak)| \leq C \delta_{B_{2n}}(a)^{\frac{1}{2}} \log(||a||)^d$ pour tous $n \in N_{2n}(F)$, $a \in A_{2n}(F)$ et $k \in GL_{2n}(\mathcal{O})$,
 \item Si $F$ est archimédien, $f$ est $C^\infty$ et il existe $d > 0$ et $C > 0$ tels que $|(R(u)f)(nak| \leq C \delta_{B_{2n}}(a)^{\frac{1}{2}} \log(||a||)^d$ pour tous $n \in N_{2n}(F)$, $a \in A_{2n}(F)$, $k \in GL_{2n}(\mathcal{O})$ et $u \in \mathcal{U}(\mathfrak{gl}_{2n}(F))$.
 \end{itemize}
 
 définir $||a||$ invariant sous la décomposition d'Iwasawa
 
 
On rappelle la majoration des fonctions tempérées sur la diagonale,
\begin{lemme}
\label{majtemp}
Soit $W \in C^w(N_{2n}(F)\backslash{GL_{2n}(F)}, \psi)$. Alors, pour tout $N \geq 1$, il existe $C > 0$ tel que
\begin{equation}
|W(bk)| \leq C\prod_{i=1}^{2n-1} (1 + |\frac{b_i}{b_{i+1}}|)^{-N}\delta_{B_{2n}}(b)^{\frac{1}{2}}\log(||b||)^d,
\end{equation}
pour tous $b \in A_{2n}(F)$ et $k \in GL_{2n}(\mathcal{O})$.
\end{lemme}

\begin{lemme}
Il existe $N$ tel que pour tous $s$ vérifiant $Re(s) > 0$ et $d > 0$, l'intégrale
\begin{equation}
\int_{A_n} \prod_{i=1}^{n-1} (1+|\frac{a_i}{a_{i+1}}|)^{-N}(1+|a_n|)^{-N}\log(||a||)^d|\det a|^s da
\end{equation}
converge absolument.
\end{lemme}

On étend la définition des intégrales $J(s, W, \phi)$ aux fonctions tempérées $W$, on montre maintenant la convergence de ces intégrales
\begin{lemme}
Pour $W \in C^w(N_{2n}(F)\backslash{GL_{2n}(F)}, \psi)$ et $\phi \in \mathcal{S}(F^n)$, l'intégrale $J(s, W, \phi)$ converge absolument pour tout $s \in \mathbb{C}$ vérifiant $Re(s) > 0$.
\end{lemme}
 
 \begin{proof}
 D'après la décomposition d'Iwasawa, on a $N_n\backslash{G_n} = A_nK_n$. Il suffit de montrer la convergence de l'intégrale
 \begin{equation}
 \int_{A_n} \int_{K_n} \int_{Lie(B_n)\backslash{M_n}} \left|W\left(\sigma \begin{pmatrix}
1 & X \\
0 & 1
\end{pmatrix}\begin{pmatrix}
ak & 0 \\
0 & ak
\end{pmatrix}\right) \phi(e_nak)\right| dX dk \left|\det a\right|^{Re(s)} \delta^{-1}(a) da.
 \end{equation}
 
 On pose $u_X = \sigma \begin{pmatrix}
1 & X \\
0 & 1
\end{pmatrix} \sigma^{-1}$, ce qui nous permet d'écrire
\begin{equation}
\sigma \begin{pmatrix}
1 & X \\
0 & 1
\end{pmatrix}\begin{pmatrix}
a & 0 \\
0 & a
\end{pmatrix} = b u_{a^{-1}Xa} \sigma,
\end{equation}
 où $b=diag(a_1,a_1,a_2,a_2,...)$. On effectue le changement de variable $X \mapsto aXa^{-1}$, l'intégrale devient alors
 \begin{equation}
\int_{A_n} \int_{K_n} \int_{Lie(B_n)\backslash{M_n}} \left|W\left(b u_X \sigma \begin{pmatrix}
k & 0 \\
0 & k
\end{pmatrix}\right)\phi(e_nak)\right|dX dk |\det a|^{Re(s)} \delta^{-2}(a) da.
 \end{equation}
 
 On écrit $u_X = n_Xt_Xk_X$ la décomposition d'Iwasawa de $u_x$ et on pose $k_\sigma = \sigma \begin{pmatrix}
k & 0 \\
0 & k
\end{pmatrix}$. Le lemme \ref{majtemp} donne alors
 \begin{equation}
 |W(bt_Xk_Xk_\sigma)| \leq C \prod_{i=1}^{2n-1} (1+  |\frac{t_jb_j}{t_{j+1}b_{j+1}}|)^{-N} \delta^{\frac{1}{2}}(bt_x)\log(||bt_X||)^d
 \end{equation}
 
 On aura besoin d'inégalités prouvées par Jacquet et Shalika \cite{jacquet-shalika} concernant les $t_j$. On dispose de la
 \begin{proposition}[Jacquet-Shalika]
 On a $|t_k| \geq 1$ lorsque $k$ est impair et $|t_k| \leq 1$ lorsque $k$ est pair. En particulier, $|\frac{t_j}{t_{j+1}}| \geq 1$ lorsque $j$ est impair et $|\frac{t_j}{t_{j+1}}| \leq 1$ lorsque $j$ est pair.
 \end{proposition}
 
 On combine alors cette proposition avec le fait que $\frac{b_j}{b_{j+1}} = 1$ lorsque $j$ est impair et $\frac{b_j}{b_{j+1}} = \frac{a_\frac{j}{2}}{a_{\frac{j}{2}+1}}$ lorsque $j$ est pair. Ce qui nous permet d'obtenir
 \begin{align}
 |W(bt_Xk_Xk_\sigma)| &\leq C 2^{-nN} \prod_{j=1, \text{j impair}}^{2n-1} |\frac{t_j}{t_{j+1}}|^{-N} \prod_{i=1}^{2n-1} (1 + |\frac{a_i}{a_{i+1}}|)^{-N} \delta^{\frac{1}{2}}(bt_x)\log(||bt_X||)^d \\
 &\leq C 2^{-nN} m(X)^{-\alpha N} \prod_{i=1}^{n-1} (1 + |\frac{a_i}{a_{i+1}}|)^{-N} \delta^{\frac{1}{2}}(bt_x)\log(||bt_X||)^d
 \end{align}
 où $m(X) = sup(1, ||X||)$, la dernière inégalité provient de la section 5.5 de Jacquet-Shalika \cite{jacquet-shalika}. D'autre part, on dispose de la majoration suivante
 \begin{equation}
 |\phi(e_nak)| \leq C'(1+|a_n|)^{-N}.
 \end{equation}
 L'intégrale est alors majorée (à une constante prés) par le produit des intégrales
 \begin{equation}
 \int_{Lie(B_n)\backslash{M_n}} m(X)^{-\alpha N} \delta^{\frac{1}{2}}(t_X)\log(||t_X||)^d dX
 \end{equation}
 et
 \begin{equation}
 \int_{A_n}  \prod_{i=1}^{n-1} (1+ |\frac{a_i}{a_{i+1}}|)^{-N} (1+|a_n|)^{-N}\log(||b||)^d|\det a|^{Re(s)} \delta_{B_{2n}}^{\frac{1}{2}}(b)\delta_{B_n}^{-2}(a) da.
 \end{equation}
 
 La première intégrale converge pour $N$ assez grand et la deuxième pour $N$ assez grand lorsque $Re(s) > 0$. On a utilisé la relation $\delta_{B_{2n}}^{\frac{1}{2}}(b) = \delta_{B_n}^2(a)$. En effet,
 \begin{align}
 \delta_{B_{2n}}(b) &= |a_1|^{1-2n}|a_1|^{3-2n}|a_2|^{5-2n}|a_2|^{7-2n}...|a_n|^{2n-3}|a_n|^{2n-1}, \\
 &= |a_1|^{4-4n}|a_2|^{12-4n}...|a_n|^{4n-4},\\
 &= \delta_{B_n}^4(a).
 \end{align}
  \end{proof}
  
 \section{Facteurs $\gamma$}
 
 Dans cette partie, on prouve l'égalité entre les facteurs $\gamma^{JS}(., \pi, \Lambda^2, \psi)$ et $\gamma^{Sh}(., \pi, \Lambda^2, \psi)$ à une constante (dépendant de $\pi$) de module 1 près.
 
 On commence à montrer cette égalité pour les facteurs $\gamma$ archimédiens. Pour le moment, les résultats connus ne nous donnent même pas l'existence du facteur $\gamma^{JS}$ dans le cas archimédien, ce sera une conséquence de la méthode de globalisation.
 
 On aura besoin d'un résultat sur la continuité du quotient $\frac{J(1-s, \rho(w_{n,n})\tilde{W}, \hat{\phi})}{J(s, W, \phi)}$ par rapport à $\pi$, on dispose du
 \begin{lemme}
 \label{cont}
 Supposons que $J(s, W, \phi) \neq 0$. Alors il existe une section $\pi' \mapsto W_{\pi'}$ et un voisinage $V \subset Temp(GL_{2n}(F))$ de $\pi$ tel que la fonction $\pi' \in V \mapsto \frac{J(1-s, \rho(w_{n,n})\tilde{W}_{\pi'}, \hat{\phi})}{J(s, W_{\pi'}, \phi)}$ soit continue.
 \end{lemme}
 
 \begin{proof}
 On utilise l'existence de bonnes sections $\pi' \mapsto W_{\pi'}$ (Beuzart-Plessis). La fonction $W \mapsto J(s, W, \phi)$ est continue, il existe donc un voisinage $V$ de $\pi$ tel que $J(s, W_{\pi'}, \phi) \neq 0$. Le quotient $\frac{J(1-s, \rho(w_{n,n})\tilde{W}_{\pi'}, \hat{\phi})}{J(s, W_{\pi'}, \phi)}$ est alors bien une fonction continue de $\pi'$ sur $V$. 
 \end{proof}
 
 \begin{proposition}
 \label{proparch}
 Soit $F = \mathbb{R}$ ou $\mathbb{C}$. Soit $\pi$ une représentation tempérée irréductible de $GL_{2n}(F)$. 
 
 Il existe une fonction méromorphe $\gamma^{JS}(s,\pi,\Lambda^2,\psi)$ telle que pour tous $s \in \mathbb{C}$, $W \in \mathcal{W}(\pi, \psi)$ et $\phi \in \mathcal{S}(F^n)$, on ait
 \begin{equation}
 \gamma^{JS}(s, \pi, \Lambda^2, \phi) J(s, W, \phi) = J(1-s, \rho(w_{n,n})\tilde{W}, \mathcal{F}_\psi(\phi)).
 \end{equation}
 
 De plus, il existe une constante $c(\pi)$ de module 1 telle que pour tout $s \in \mathbb{C}$,
 \begin{equation}
 \gamma^{JS}(s, \pi, \Lambda^2, \psi) = c(\pi)\gamma^{Sh}(s, \pi, \Lambda^2, \psi).
 \end{equation}
 \end{proposition}
 
 \begin{proof}
 Soit $k$ un corps de nombres, on suppose que $k$ a une seule place archimédienne, elle est réelle (respectivement complexe) lorsque $F=\mathbb{R}$ (respectivement $F=\mathbb{C}$); par exemple, $k=\mathbb{Q}$ si $F=\mathbb{R}$ et $k=\mathbb{Q}(i)$ si $F=\mathbb{C}$. Soit $v \neq v'$ deux places non archimédiennes distinctes, soit $U \subset Temp(GL_{2n}(F))$ un ouvert contenant $\pi$. On choisit un caractère $\psi_\mathbb{A}$ de $\mathbb{A}_K/K$ tel que $(\psi_\mathbb{A})_\infty = \psi$.
 
 D'après la proposition \ref{globalisation}, il existe une représentation automorphe cuspidale irréductible $\Pi$ telle que $\Pi_{\infty} \in U$ et $\Pi_w$ soit non ramifiée pour toute place non archimédienne $w \neq v$.
 
 On choisit maintenant des fonctions de Whittaker $W_w$ et des fonctions de Schwartz $\phi_w$ dans le but d'appliquer l'équation fonctionnelle globale. Pour $w \not\in \{\infty, v\}$, on prend les fonctions "non ramifiées" qui apparaissent dans la proposition \ref{calculnr}. Pour $w = \infty$ ou $v$, on fait un choix, d'après la proposition \ref{nonzero}, tel que $J(s, W_w, \phi_w) \neq 0$. On pose alors
 $$W = \prod_w W_w \quad \text{et} \quad \Phi  = \prod_w \phi_w.$$
 
 D'après la proposition \ref{funcglob}, on a
 \begin{equation}
 \label{jacquet-shalika}
 \begin{split}
 &J(s, W_\infty, \phi_\infty)J(s, W_v, \phi_v)L^S(s, \Pi, \Lambda^2) \\
 &= J(1-s, \rho(w_{n,n})\tilde{W}_\infty, \mathcal{F}_\psi(\phi_\infty))J(1-s, \rho(w_{n,n})\tilde{W}_v, \mathcal{F}_{(\psi_\mathbb{A})_v}(\phi_v))L^S(1-s, \tilde{\Pi}, \Lambda^2),
 \end{split}
 \end{equation}
 où $S = \{\infty, v\}$ et $L^S(s, \Pi, \Lambda^2) = \prod_{w \neq \infty,v} L(s, \Pi_w, \Lambda^2)$ est la fonction L partielle. D'autre part, les facteurs $\gamma$ de Shahidi vérifient une relation similaire,
 \begin{equation}
 \label{shahidi}
 L^S(s, \Pi, \Lambda^2) = \gamma^{Sh}(s, \Pi_\infty, \Lambda^2, \psi)\gamma^{Sh}(s, \Pi_v, \Lambda^2, (\psi_\mathbb{A})_v)L^S(1-s, \tilde{\Pi}, \Lambda^2).
 \end{equation}
 
 Le quotient de (\ref{jacquet-shalika}) et (\ref{shahidi}), en utilisant la proposition \ref{funcloc} sur le facteur en $\Pi_v$, donne
 \begin{equation}
 \frac{J(1-s, \rho(w_{n,n})\tilde{W}_\infty, \mathcal{F}_\psi(\phi_\infty))}{J(s, W_\infty, \phi_\infty)\gamma^{Sh}(s, \Pi_\infty, \Lambda^2, \psi)} \frac{\gamma^{JS}(s, \Pi_v, \Lambda^2, (\psi_\mathbb{A})_v)}{\gamma^{Sh}(s, \Pi_v, \Lambda^2, (\psi_\mathbb{A})_v)} = 1.
 \end{equation}
 
 Ce qui prouve la première partie de la proposition pour $\Pi_\infty$, l'existence du facteur $\gamma^{JS}(s, \Pi_\infty, \Lambda^2, \psi)$.
 
 On choisit maintenant pour $U$ une base de voisinage contenant $\pi$, en utilisant le lemme \ref{cont} et la continuité des facteurs $\gamma$ de Shahidi, on en déduit que $\frac{J(1-s, \rho(w_{n,n})\tilde{W}, \mathcal{F}_\psi(\phi))}{J(s, W, \phi)}$
 est une fonction méromorphe indépendante de $W$ et de $\phi$, que l'on note $\gamma^{JS}(s, \pi, \Lambda^2, \psi)$, qui est le produit de $\gamma^{Sh}(s, \pi, \Lambda^2, \psi)$ et d'une fonction, que l'on note $R(s)$, qui est limite de fractions rationnelles en $q_v^s$; donc $R$ est une fonction périodique de période $\frac{2i\pi}{\log q_v}$.
 
  On réutilisant notre raisonnement en la place $v'$, on voit que $R$ est aussi périodique de période $\frac{2i\pi}{\log q_{v'}}$; donc est constante. Ce qui nous permet de voir qu'il existe une constante $c(\pi)=R$ telle que
 \begin{equation}
 \gamma^{JS}(s, \pi, \Lambda^2, \psi) = c(\pi)\gamma^{Sh}(s, \pi, \Lambda^2, \psi).
 \end{equation}
 
 Il ne nous reste plus qu'à montrer que la constante $c(\pi)$ est de module 1. Reprenons l'équation fonctionnelle locale archimédienne,
 \begin{equation}
 \label{funcarch}
 \gamma^{JS}(s, \pi, \Lambda^2, \psi) J(s, W, \phi) = J(1-s, \rho(w_{n,n})\tilde{W}, \mathcal{F}_\psi(\phi)).
 \end{equation}
 
 On utilise maintenant l'équation fonctionnelle sur la représentation $\tilde{\pi}$ pour transformer le facteur $J(1-s, \rho(w_{n,n})\tilde{W}, \mathcal{F}_\psi(\phi))$, ce qui nous donne
 \begin{equation}
 \gamma^{JS}(s, \pi, \Lambda^2, \psi) J(s, W, \phi) = \frac{J(s, W, \mathcal{F}_{\bar{\psi}}(\mathcal{F}_\psi(\phi)))}{\gamma^{JS}(1-s, \tilde{\pi}, \Lambda^2, \bar{\phi})}.
 \end{equation}
 
 Puisque $\mathcal{F}_{\bar{\psi}}(\mathcal{F}_\psi(\phi)) = \phi$, on obtient donc la relation 
 \begin{equation}
 \gamma^{JS}(s, \pi, \Lambda^2, \psi)\gamma^{JS}(1-s, \tilde{\pi}, \Lambda^2, \bar{\phi}) = 1.
 \end{equation}
 
 D'autre part, en conjuguant l'équation \ref{funcarch}, on obtient
 \begin{equation}
 \overline{\gamma^{JS}(s, \pi, \Lambda^2, \psi)} = \gamma^{JS}(\bar{s}, \bar{\pi}, \Lambda^2, \bar{\psi}).
 \end{equation}
 
 Comme $\pi$ est tempérée, $\pi$ est unitaire, donc $\tilde{\pi} \simeq \bar{\pi}$. On en déduit, pour $s = \frac{1}{2}$,
 \begin{equation}
 |\gamma^{JS}(\frac{1}{2}, \pi, \Lambda^2, \psi)|^2=1.
 \end{equation}
 
 D'autre part, le facteur $\gamma$ de Shahidi vérifie aussi $|\gamma^{JS}(\frac{1}{2}, \pi, \Lambda^2, \psi)|^2=1$; on en déduit donc que $c(\pi)$ est bien de module 1.
 \end{proof}
 
 \begin{proposition}
 Supposons que $F$ est un corps $p$-adique. Soit $\pi$ une représentation tempérée irréductible de $GL_{2n}(F)$. 
 
 Le facteur $\gamma^{JS}(s,\pi,\Lambda^2,\psi)$ est défini par la proposition \ref{funcloc}. Alors il existe une constante $c(\pi)$ de module 1 telle que pour tout $s \in \mathbb{C}$,
 \begin{equation}
 \gamma^{JS}(s, \pi, \Lambda^2, \psi) = c(\pi)\gamma^{Sh}(s, \pi, \Lambda^2, \psi).
 \end{equation}
 \end{proposition}
 
 \begin{proof}
 D'après le lemme \ref{corpsglobal}, il existe un corps de nombres $k$ et une place $v_0$ telle que $k_{v_0} = F$, où $v_0$ est l'unique place de $k$ au dessus de $p$. Soit $v,v'$ deux places distinctes non archimédiennes et différentes de $v_0$. Soit $U \subset Temp(GL_{2n}(F))$ un ouvert contenant $\pi$. On choisit un caractère $\psi_\mathbb{A}$ de $\mathbb{A}_k/k$ tel que $(\psi_\mathbb{A})_{v_0} = \psi$.
 
 D'après la proposition \ref{globalisation}, il existe une représentation automorphe cuspidale irréductible $\Pi$ telle que $\Pi_{v_0} \in U$ et $\Pi_w$ soit non ramifiée pour toute place non archimédienne $w \neq v$.
 
 Pour $w = v_0,v$ ou une place archimédienne, on choisit d'après la proposition \ref{nonzero}, des fonctions de Whittaker $W_w$ et des fonctions de Schwartz $\phi_w$ telles que $J(s, W_w, \phi_w) \neq 0$. Pour les places non ramifiées, on choisit les fonctions "non ramifiées" de la proposition \ref{calculnr}. On pose alors
 $$W = \prod_w W_w \quad \text{et} \quad \Phi  = \prod_w \phi_w.$$
 
 D'après l'équation fonctionnelle globale (proposition \ref{funcglob}), on a
 \begin{equation}
 \begin{split}
 &\prod_{w \in \{v,v_0,v_\infty\}} J(s, W_w, \phi_w) L^S(s, \Pi, \Lambda^2)\\
 &= \prod_{w \in \{v,v_0,v_\infty\}} J(1-s, \rho(w_{n,n})\tilde{W}_w, \mathcal{F}_{(\psi_\mathbb{A})_w}(\phi_w)) L^S(1-s, \tilde{\Pi}, \Lambda^2),
 \end{split}
 \end{equation}
 où $v_\infty$ décrit les places archimédiennes, $S = \{v_\infty\} \cup \{v, v_0\}$ et $L^S(s, \Pi, \Lambda^2)$ est la fonction L partielle. Les facteurs $\gamma$ de Shahidi vérifient
 \begin{equation}
 L^S(s, \Pi, \Lambda^2) = \prod_{w \in S}\gamma^{Sh}(s, \Pi_w, \Lambda^2, (\psi_\mathbb{A})_w) L^S(1-s, \tilde{\Pi}, \Lambda^2).
 \end{equation}
 
 En utilisant les propositions \ref{funcloc} et \ref{proparch}, on obtient donc la relation
 \begin{equation}
 \prod_{v_\infty} c(\Pi_{v_\infty}) \frac{\gamma^{JS}(s, \Pi_v, \Lambda^2, (\psi_\mathbb{A})_v)}{\gamma^{Sh}(s, \Pi_v, \Lambda^2, (\psi_\mathbb{A})_v)}\frac{\gamma^{JS}(s, \Pi_{v_0}, \Lambda^2, \psi)}{\gamma^{Sh}(s, \Pi_{v_0}, \Lambda^2, \psi)} = 1.
 \end{equation}
 
 Le reste du raisonnement est maintenant identique à la fin de la preuve de la proposition \ref{proparch}. Par continuité, le quotient $|\frac{\gamma^{JS}(s, \pi, \Lambda^2, \psi)}{\gamma^{Sh}(s, \pi, \Lambda^2, \psi)}|$ est une fonction périodique de période $\frac{2i\pi}{\log q_v}$. En appliquant le même raisonnement en la place $v'$, on obtient que c'est une constante. En évaluant $\gamma^{JS}(s, \pi, \Lambda^2, \psi)$ en $s=\frac{1}{2}$, on montre que cette constante est $1$.
 \end{proof}
 
 \bibliographystyle{siam}
\bibliography{carre-exterieur}

\end{document}
