\documentclass{amsart}

%\usepackage[utf8]{inputenc}
\usepackage[T1]{fontenc}
\usepackage[francais]{babel}
\usepackage{bbm}
\usepackage{amssymb}
\usepackage{amsmath}
\usepackage{amsthm}
\usepackage{mathtools}
\usepackage{hyperref}
\usepackage{graphics}
\usepackage{enumerate}

\usepackage{eulervm}

\newtheorem{proposition}{Proposition}[section]
\newtheorem{corollaire}{Corollaire}[section]
\newtheorem{propriete}{Propriété}[section]
\newtheorem{definition}{Définition}[section]
\newtheorem{theoreme}{Théorème}[section]
\newtheorem{lemme}{Lemme}[section]

\DeclareMathOperator{\Hom}{\mathnormal{Hom}}
\DeclareMathOperator{\Ext}{\mathnormal{Ext}}

\begin{document}

\title{Formule de Plancherel}
%\date{\today}
\maketitle 
 
Pour $W \in \mathcal{C}^w(N_{2n} \backslash G_{2n})$, on note
\begin{equation}
\label{beta}
\beta(W) = \int_{H^P_n \cap N_{2n} \backslash H^P_n} W(\xi_p) \theta(\xi_p)^{-1} d\xi_p.
\end{equation}

\begin{lemme}
L'intégrale \ref{beta} est absolument convergente. La forme linéaire $W \in \mathcal{W}(\pi, \psi) \mapsto \beta(W)$ est continue et c'est un élément de $\Hom_{H_n}(\mathcal{W}(\pi, \psi), \theta)$.
\end{lemme}

\begin{proof}
\end{proof}

\begin{proposition}
\label{constbeta}
Soit $\sigma \in Temp(SO(2n+1))$, on pose $\pi = T(\sigma)$ le transfert de $\sigma$ dans $Temp(G_{2n})$. La forme linéaire $\widetilde{W} \in \mathcal{W}(\widetilde{\pi}, \psi^{-1}) \mapsto \beta(\widetilde{W})$ est un élément de $\Hom_{H_n}(\mathcal{W}(\widetilde{\pi}, \psi^{-1}), \theta)$. On identifie $\mathcal{W}(\pi, \psi)$ et $\mathcal{W}(\widetilde{\pi}, \psi^{-1})$ par l'isomorphisme $W \mapsto \widetilde{W}$. Il existe un signe $c_\beta(\sigma) = c_\beta(\pi)$ tel que
\begin{equation}
\beta(\widetilde{W}) = c_\beta(\sigma)\beta(W),
\end{equation}
pour tout $W \in \mathcal{W}(\pi, \psi)$.
\end{proposition}

\begin{proof}
En effet, $\Hom_{H^P_n}(\mathcal{W}(\pi, \psi), \theta)$ est de dimension au plus 1 (Matringe, Prop 4.3). De plus, $\pi$ est le transfert de $\sigma$ donc $\widetilde{\pi} \simeq \pi$. On en déduit l'existence de $c_\beta(\pi) \in \mathbb{C}$ qui vérifie $c_\beta(\widetilde{\pi})c_\beta(\pi) = 1$ donc $c_\beta(\pi)$ est un signe.
\end{proof}

\begin{lemme}
\label{zetabeta}
Soit $\sigma \in Temp(SO(2n+1))$ et $\pi = T(\sigma)$. Alors
\begin{equation}
J(1, \widetilde{W}, \widehat{\phi}) = \phi(0)c_\beta(\sigma)\beta(W),
\end{equation}
pour tous $W \in \mathcal{W}(\pi, \psi)$ et $\phi \in \mathcal{S}(F^n)$.
\end{lemme}

\begin{proof}
En effet, on a
\begin{equation}
\begin{split}
J(1, \widetilde{W}, \widehat{\phi}) &= \int_{N_n \backslash G_n} \int_{Lie(B_n) \backslash M_n}\widetilde{W}\left(\sigma\begin{pmatrix}
1 & X \\
0 & 1
\end{pmatrix} \begin{pmatrix}
g & 0 \\
0 & g
\end{pmatrix} \sigma^{-1}\right) \psi(-Tr(X)) dX \widehat{\phi}(e_ng) |\det g| dg \\
&= \int_{P_n \backslash G_n} \int_{N_n \backslash P_n} \int_{Lie(B_n) \backslash M_n}\widetilde{W}\left(\sigma\begin{pmatrix}
1 & X \\
0 & 1
\end{pmatrix} \begin{pmatrix}
ph & 0 \\
0 & ph
\end{pmatrix} \sigma^{-1} \right) \psi(-Tr(X)) dX dp \widehat{\phi}(e_nh) |\det h| dh.
\end{split}
\end{equation}

On remarque que l'on a
\begin{equation}
\begin{split}
\int_{N_n \backslash P_n} \int_{Lie(B_n) \backslash M_n}\widetilde{W}\left(\sigma\begin{pmatrix}
1 & X \\
0 & 1
\end{pmatrix} \begin{pmatrix}
ph & 0 \\
0 & ph
\end{pmatrix} \sigma^{-1}\right) \psi(-Tr(X)) dX dp &= \beta\left(R\left(\sigma \begin{pmatrix}
h & 0 \\
0 & h
\end{pmatrix} \sigma^{-1}\right) \widetilde{W}\right) \\
&= \beta(\widetilde{W}),
\end{split}
\end{equation}
puisque $\beta$ est $H_n$-invariant. De plus,
\begin{equation}
\begin{split}
\int_{P_n \backslash G_n}  \widehat{\phi}(e_nh) |\det h| dh &= \int_{F^n} \widehat{\phi}(x) dx \\
&= \phi(0).
\end{split}
\end{equation}

On conclut grâce à la proposition \ref{constbeta}.
\end{proof}

Soit $f \in \mathcal{S}(G_{2n})$ et $\pi \in Temp(G_{2n})$, on pose $W_{f, \pi} = W_{f_\pi}$.

\begin{lemme}
\label{limitezeta}
Pour $W \in \mathcal{S}(Z_{2n} N_{2n} \backslash G_{2n})$ et $\phi \in \mathcal{S}(F^n)$, on a
\begin{equation}
\lim_{s\rightarrow 0^+} \gamma(ns, 1, \psi) J(s, W, \phi) = \phi(0) \int_{Z_{2n}(H_n \cap N_{2n}) \backslash H_n} W(\xi) \theta(\xi)^{-1} d\xi.
\end{equation}
\end{lemme}

\begin{proof}
On a
\begin{equation}
\begin{split}
\gamma(ns, 1, \psi) J(s, W, \phi) &= \int_{Z_n \backslash A_n} \int_{K_n} \int_{Lie(B_n) \backslash M_n}\widetilde{W}\left(\sigma\begin{pmatrix}
1 & X \\
0 & 1
\end{pmatrix} \begin{pmatrix}
ak & 0 \\
0 & ak
\end{pmatrix} \sigma^{-1}\right) \\
&\psi(-Tr(X)) dX \gamma(ns, 1, \psi) \int_{Z_n}\phi(e_nzk) |\det z|^s dz dk |\det a|^s \delta_{B_n}(a)^{-1} da
\end{split}
\end{equation}

De plus,
\begin{equation}
\gamma(ns, 1, \psi) \int_{Z_n} \phi(e_n zk) |\det z|^s ds = \int_{F^*} \widehat{\phi_k}(x)|x|^{1-ns} dx,
\end{equation}
où l'on a posé $\phi_k(x) = \phi(xe_nk)$ pour tous $x \in F$ et $k \in K_n$. Ce qui nous donne
\begin{equation}
\lim_{s \rightarrow 0+} \gamma(ns, 1, \psi)\int_{Z_n} \phi(e_nzk) |\det z|^s dz = \int_{F} \widehat{\phi_k}(x)dx = \phi(0).
\end{equation}

On en déduit que
\begin{equation}
\begin{split}
\lim_{s \rightarrow 0^+}\gamma(ns, 1, \psi) J(s, W, \phi) &= \phi(0)\int_{Z_n \backslash A_n} \int_{K_n} \int_{Lie(B_n) \backslash M_n}\widetilde{W}\left(\sigma\begin{pmatrix}
1 & X \\
0 & 1
\end{pmatrix} \begin{pmatrix}
ak & 0 \\
0 & ak
\end{pmatrix} \sigma^{-1}\right) \\
&\psi(-Tr(X)) dX  dk \delta_{B_n}(a)^{-1} da,
\end{split}
\end{equation}
ce qui nous permet de conclure.
\end{proof}

\begin{corollaire}[de la limite spectrale]
\label{corolim}
Soit $f \in \mathcal{S}(G_{2n})$ et $g \in G_{2n}$, alors
\begin{equation}
\int_{H_n \cap N_{2n} \backslash H_n} W_f(g, \xi) \theta(\xi)^{-1} d\xi = \int_{Temp(SO(2n+1))/Stab} \beta(W_{f,T(\sigma)}(g,.)) \frac{\gamma^*(0, \sigma, Ad, \psi)}{|S_\sigma|} c(T(\sigma)) c_\beta(\sigma) d\sigma.
\end{equation}
\end{corollaire}

\begin{proof}
On peut supposer que $g = 1$ en remplaçant $f$ par $L(g)f$. On pose $\widetilde{f}(g) = \int_{Z_n} f(zg) dz$, alors $\widetilde{f} \in PG_{2n}$. On a donc
\begin{equation}
\int_{H_n \cap N_{2n} \backslash H_n} W_f(1, \xi) \theta(\xi)^{-1} d\xi = \int_{Z_{2n}(H_n \cap N_{2n}) \backslash H_n} W_{\widetilde{f}}(1, \xi) \theta(\xi)^{-1} d\xi.
\end{equation}

On choisit $\phi \in \mathcal{S}(F^n)$ tel que $\phi(0) = 1$. Comme $\widetilde{f}_\pi = f_\pi$ pour tout $\pi \in Temp(PG_{2n})$, d'après le lemme \ref{limitezeta}, on a
\begin{equation}
\begin{split}
\int_{Z_{2n}(H_n \cap N_{2n}) \backslash H_n} W_{\widetilde{f}}(1, \xi) \theta(\xi)^{-1} d\xi &= \lim_{s\rightarrow 0^+} n\gamma(s, 1, \psi) J(s, W_{\widetilde{f}}(1, .), \phi) \\
&= \lim_{s\rightarrow 0^+} n\gamma(s, 1, \psi) \int_{Temp(PG_{2n})}J(s, W_{f, \pi}(1, .), \phi) d\mu_{PG_{2n}}(\pi).
\end{split}
\end{equation}

D'après l'équation fonctionnelle, on a
\begin{equation}
\begin{split}
&\int_{H_n \cap N_{2n} \backslash H_n} W_f(1, \xi) \theta(\xi)^{-1} d\xi = \\
&\lim_{s\rightarrow 0^+} n\gamma(s, 1, \psi) \int_{Temp(PG_{2n})}J(1-s, \widetilde{W_{f, \pi}(1, .)}, \widehat{\phi}) c(\pi) \gamma(s, \pi, \Lambda^2, \psi)^{-1} d\mu_{PG_{2n}}(\pi).
\end{split}
\end{equation}

D'apres (ref limite spectrale), cette dernière limite est égale à
\begin{equation}
\int_{Temp(SO(2n+1)/Stab} J(1, \widetilde{W_{f, T(\sigma)}(1,.)}, \widehat{\phi}) c(T(\sigma)) \frac{\gamma^*(0, \sigma, Ad, \psi)}{|S_\sigma|} d\sigma.
\end{equation}

On conclut grâce au lemme \ref{zetabeta}.
\end{proof}

\section{Formule de Plancherel}

On note $Y_n = H_n \backslash G_{2n}$. On dispose d'une surjection $f \in \mathcal{S}(G_{2n}) \mapsto \varphi_f \in \mathcal{S}(Y_n, \theta)$ avec
\begin{equation}
\varphi_f(x) = \int_{H_n} f(hx) \theta(h)^{-1} dh,
\end{equation}
pour tous $x \in Y_n$. 

%Pour $\pi \in Temp(G_{2n})$ et $f_1,f_2 \in \mathcal{S}(G_{2n})$, on pose
%\begin{equation}
%(f_1,f_2)_{Y_n, \pi} = \sum_{W \in \mathcal{B}(\pi, \psi)} \beta(R(f_1)W)\overline{\beta(R(f_2)W)},
%\end{equation}
%où $\mathcal{B}(\pi, \psi)$ est une base orthonormée de $\mathcal{W}(\pi, \psi)$.

\begin{theoreme}
Soit $\varphi_1, \varphi_2 \in \mathcal{S}(Y_n)$, il existe $f_1, f_2 \in \mathcal{S}(G_{2n})$ tel que $\varphi_i = \varphi_{f_i}$ pour $i = 1,2$. On a
\begin{equation}
\label{psf}
(\varphi_1, \varphi_2)_{L^2(Y_n)} = \int_{H_n} f(h) \theta(h)^{-1} dh,
\end{equation}
où $f = f_1 * f_2^{*}$, on note $f_2^*(g) = \overline{f_2(g^{-1})}$. On pose
\begin{equation}
(\varphi_1, \varphi_2)_{Y_n, \pi} = \int_{H^P_n \cap N_{2n} \backslash H^P_n} \beta\left(W_{f,\pi}(\xi_p,.)\right) \theta(\xi_p)^{-1} d\xi_p,
\end{equation}
pour tous $\pi \in Temp(G_{2n})$. Alors on a
\begin{equation}
(\phi_1, \phi_2)_{L^2(Y_n)} = \int_{Temp(SO(2n+1))/Stab} (\varphi_1, \varphi_2)_{Y_n, T(\sigma)} \frac{|\gamma^*(0, \sigma, Ad, \psi)|}{|S_\sigma|}d\sigma.
\end{equation}
\end{theoreme}

\begin{proof}
On a
\begin{equation}
(\varphi_1, \varphi_2)_{L^2(Y_n)} = \int_{Y_n} \int_{H_n \times H_n} f_1(h_1 y) \overline{f_2(h_2 y)} \theta(h_1)^{-1} \theta(h_2) dh_1 dh_2 dy.
\end{equation}

L'integrale est absoluement convergente. En effet,
\begin{equation}
(y,h_1,h_2) \in \mathcal{Y}_n \times H_n \times H_n \mapsto f_1(h_1 y) \overline{f_2(h_2 y)}
\end{equation}
est a support compact, ou $\mathcal{Y}_n$ est un systeme de representant de $Y_n$.
On effectue le changement de variable $h_1 \mapsto h_1h_2$ et on combine les integrales selon $y$ et $h_2$ en une integrale sur $G_{2n}$. Ce qui donne
\begin{equation}
(\varphi_1, \varphi_2)_{L^2(Y_n)} = \int_{G_{2n}} \int_{H_n} f_1(h_1 y) \overline{f_2(y)} \theta(h_1)^{-1} dh_1 dy,
\end{equation}
qui est bien la relation \ref{psf}.


D'apres [unfolding] et \ref{corolim}, on a
\begin{equation}
\label{intfin}
\begin{split}
\int_{H_n} f(h) \theta(h)^{-1} dh = &\int_{H_n \cap N_{2n} \backslash H^P_n} \int_{Temp(SO(2n+1))/Stab} \beta\left(W_{f,T(\sigma)}(\xi_p,.)\right) \\
& \frac{\gamma^*(0, \sigma, Ad, \psi)}{|S_\sigma|}c(T(\sigma))c_\beta(\sigma) d\sigma d\xi_p.
\end{split}
\end{equation}

\begin{lemme}
La fonction $(\xi_p, \sigma) \mapsto \beta\left(W_{f,T(\sigma)}(\xi_p,.)\right)$ est a support compact, l'integrale \ref{intfin} est donc absolument convergente.
\end{lemme}

\begin{proof}

\end{proof}

On echange les integrales pour obtenir
\begin{equation}
\int_{Temp(SO(2n+1))/Stab} (\varphi_1, \varphi_2)_{Y_n, T(\sigma)} \frac{\gamma^*(0, \sigma, Ad, \psi)}{|S_\sigma|} c(T(\sigma))c_\beta(\sigma)d\sigma.
\end{equation}

Montrer l'independance par rapport au choix de $f_1, f_2$.

Pour finir, [beuzart-plessis, prop 4.1.1] nous dit que les formes sesquilineaires $(\varphi_1, \varphi_2) \mapsto (\varphi_1, \varphi_2)_{Y_n, T(\sigma)} \frac{\gamma^*(0, \sigma, Ad, \psi)}{|S_\sigma|} c(T(\sigma))c_\beta(\sigma)$ sont automatiquement definies positives. On en deduit que 
\begin{equation}
\gamma^*(0, \sigma, Ad, \psi) c(T(\sigma))c_\beta(\sigma) = |\gamma^*(0, \sigma, Ad, \psi)|.
\end{equation}
\end{proof}





\end{document}
