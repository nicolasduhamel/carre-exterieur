\documentclass{amsart}

%\usepackage[utf8]{inputenc}
\usepackage[T1]{fontenc}
\usepackage[francais]{babel}
\usepackage{bbm}
\usepackage{amssymb}
\usepackage{amsmath}
\usepackage{amsthm}
\usepackage{mathtools}
\usepackage{hyperref}
\usepackage{graphics}
\usepackage{enumerate}

\usepackage{eulervm}

\newtheorem{proposition}{Proposition}[section]
\newtheorem{corollaire}{Corollaire}[section]
\newtheorem{propriete}{Propri�t�}[section]
\newtheorem{definition}{D�finition}[section]
\newtheorem{theoreme}{Th�or�me}[section]
\newtheorem{lemme}{Lemme}[section]

\DeclareMathOperator{\Hom}{\mathnormal{Hom}}
\DeclareMathOperator{\Ext}{\mathnormal{Ext}}

\begin{document}

\title{Formule de Plancherel}
%\date{\today}
\maketitle 
 
Pour $W \in \mathcal{C}^w(N_{2n} \backslash G_{2n})$, on note
\begin{equation}
\label{beta}
\beta(W) = \int_{H^P_n \cap N_{2n} \backslash H^P_n} W(\xi_p) \theta(\xi_p)^{-1} d\xi_p.
\end{equation}

\begin{lemme}
L'int�grale \ref{beta} est absolument convergente.
\end{lemme}

\begin{proof}
\end{proof}

\begin{proposition}
La forme lin�aire $W \in \mathcal{W}(\pi, \psi) \mapsto \beta(W)$ est un �l�ment de $\Hom_{H_n}(\mathcal{W}(\pi, \psi), \theta)$.
\end{proposition}

\begin{proof}
\end{proof}

\begin{proposition}
\label{constbeta}
Soit $\sigma \in Temp(SO(2n+1))$, on pose $\pi = T(\sigma)$ le transfert de $\sigma$ dans $Temp(G_{2n})$. La forme lin�aire $\widetilde{W} \in \mathcal{W}(\widetilde{\pi}, \psi^{-1}) \mapsto \beta(\widetilde{W})$ est un �l�ment de $\Hom_{H_n}(\mathcal{W}(\widetilde{\pi}, \psi^{-1}), \theta)$. On identifie $\mathcal{W}(\pi, \psi)$ et $\mathcal{W}(\widetilde{\pi}, \psi^{-1})$ par l'isomorphisme $W \mapsto \widetilde{W}$. Il existe un signe $c_\beta(\sigma) = c_\beta(\pi)$ tel que
\begin{equation}
\beta(\widetilde{W}) = c_\beta(\sigma)\beta(W),
\end{equation}
pour tout $W \in \mathcal{W}(\pi, \psi)$.
\end{proposition}

\begin{proof}
En effet, $\Hom_{H_n}(\mathcal{W}(\pi, \psi), \theta)$ est de dimension au plus 1 (preuve, r�f�rence). De plus, $\pi$ est le transfert de $\sigma$ donc $\widetilde{\pi} \simeq \pi$. On en d�duit l'existence de $c(\pi) \in \mathbb{C}$ qui v�rifie $c(\widetilde{\pi})c(\pi) = 1$ donc $c(\pi)$ est un signe.
\end{proof}

\begin{lemme}
Soit $\sigma \in Temp(SO(2n+1))$ et $\pi = T(\sigma)$. Alors
\begin{equation}
J(1, \widetilde{W}, \widehat{\phi}) = \phi(0)c_\beta(\sigma)\beta(W),
\end{equation}
pour tous $W \in \mathcal{W}(\pi, \psi)$ et $\phi \in \mathcal{S}(F^n)$.
\end{lemme}

\begin{proof}
En effet, on a
\begin{equation}
\begin{split}
J(1, \widetilde{W}, \widehat{\phi}) &= \int_{N_n \backslash G_n} \int_{Lie(B_n) \backslash M_n}\widetilde{W}\left(\sigma\begin{pmatrix}
1 & X \\
0 & 1
\end{pmatrix} \begin{pmatrix}
g & 0 \\
0 & g
\end{pmatrix} \sigma^{-1}\right) \psi(-Tr(X)) dX \widehat{\phi}(e_ng) |\det g| dg \\
&= \int_{P_n \backslash G_n} \int_{N_n \backslash P_n} \int_{Lie(B_n) \backslash M_n}\widetilde{W}\left(\sigma\begin{pmatrix}
1 & X \\
0 & 1
\end{pmatrix} \begin{pmatrix}
ph & 0 \\
0 & ph
\end{pmatrix} \sigma^{-1} \right) \psi(-Tr(X)) dX dp \widehat{\phi}(e_nh) |\det h| dh.
\end{split}
\end{equation}

On remarque que l'on a
\begin{equation}
\begin{split}
\int_{N_n \backslash P_n} \int_{Lie(B_n) \backslash M_n}\widetilde{W}\left(\sigma\begin{pmatrix}
1 & X \\
0 & 1
\end{pmatrix} \begin{pmatrix}
ph & 0 \\
0 & ph
\end{pmatrix} \sigma^{-1}\right) \psi(-Tr(X)) dX dp &= \beta\left(R\left(\sigma \begin{pmatrix}
h & 0 \\
0 & h
\end{pmatrix} \sigma^{-1}\right) \widetilde{W}\right) \\
&= \beta(\widetilde{W}),
\end{split}
\end{equation}
puisque $\beta$ est $H_n$-invariant. De plus,
\begin{equation}
\begin{split}
\int_{P_n \backslash G_n}  \widehat{\phi}(e_nh) |\det h| dh &= \int_{F^n} \widehat{\phi}(x) dx \\
&= \phi(0).
\end{split}
\end{equation}

On conclut gr�ce � la proposition \ref{constbeta}.
\end{proof}

Soit $f \in \mathcal{S}(G_{2n})$ et $\pi \in Temp(G_{2n})$, on pose $W_{f, \pi} = W_{f_\pi}$.

\begin{lemme}
\label{limitezeta}
Pour $W \in \mathcal{S}(Z_{2n} N_{2n} \backslash G_{2n})$ et $\phi \in \mathcal{S}(F^n)$, on a
\begin{equation}
\lim_{s\rightarrow 0^+} \gamma(ns, 1, \psi) J(s, W, \phi) = \phi(0) \int_{Z_{2n}(H_n \cap N_{2n}) \backslash H_n} W(\xi) \theta(\xi)^{-1} d\xi.
\end{equation}
\end{lemme}

\begin{proof}
On a
\begin{equation}
\begin{split}
\gamma(ns, 1, \psi) J(s, W, \phi) &= \int_{Z_n \backslash A_n} \int_{K_n} \int_{Lie(B_n) \backslash M_n}\widetilde{W}\left(\sigma\begin{pmatrix}
1 & X \\
0 & 1
\end{pmatrix} \begin{pmatrix}
ak & 0 \\
0 & ak
\end{pmatrix} \sigma^{-1}\right) \\
&\psi(-Tr(X)) dX \gamma(ns, 1, \psi) \int_{Z_n}\phi(e_nzk) |\det z|^s dz dk |\det a|^s \delta_{B_n}(a)^{-1} da
\end{split}
\end{equation}

De plus,
\begin{equation}
\gamma(ns, 1, \psi) \int_{Z_n} \phi(e_n zk) |\det z|^s ds = \int_{F^*} \widehat{\phi_k}(x)|x|^{1-ns} dx,
\end{equation}
o� l'on a pos� $\phi_k(x) = \phi(xe_nk)$ pour tous $x \in F$ et $k \in K_n$. Ce qui nous donne
\begin{equation}
\lim_{s \rightarrow 0+} \gamma(ns, 1, \psi)\int_{Z_n} \phi(e_nzk) |\det z|^s dz = \int_{F} \widehat{\phi_k}(x)dx = \phi(0).
\end{equation}

On en d�duit que
\begin{equation}
\begin{split}
\lim_{s \rightarrow 0^+}\gamma(ns, 1, \psi) J(s, W, \phi) &= \phi(0)\int_{Z_n \backslash A_n} \int_{K_n} \int_{Lie(B_n) \backslash M_n}\widetilde{W}\left(\sigma\begin{pmatrix}
1 & X \\
0 & 1
\end{pmatrix} \begin{pmatrix}
ak & 0 \\
0 & ak
\end{pmatrix} \sigma^{-1}\right) \\
&\psi(-Tr(X)) dX  dk \delta_{B_n}(a)^{-1} da,
\end{split}
\end{equation}
ce qui nous permet de conclure.
\end{proof}

\begin{corollaire}[de la limite spectrale]
Soit $f \in \mathcal{S}(G_{2n})$ et $g \in G_{2n}$, alors
\begin{equation}
\int_{H_n \cap N_{2n} \backslash H_n} W_f(g, \xi) \theta(\xi)^{-1} d\xi = \int_{Temp(SO(2n+1))/Stab} \beta(W_{f,T(\sigma)}(g,.)) \frac{\gamma^*(0, \sigma, Ad, \psi)}{|S_\sigma|} c(T(\sigma)) c_\beta(\sigma) d\sigma.
\end{equation}
\end{corollaire}

\begin{proof}
On peut supposer que $g = 1$ en rempla�ant $f$ par $L(g)f$. On pose $\widetilde{f}(g) = \int_{Z_n} f(zg) dz$, alors $\widetilde{f} \in PG_{2n}$. On a donc
\begin{equation}
\int_{H_n \cap N_{2n} \backslash H_n} W_f(1, \xi) \theta(\xi)^{-1} d\xi = \int_{Z_{2n}(H_n \cap N_{2n}) \backslash H_n} W_{\widetilde{f}}(1, \xi) \theta(\xi)^{-1} d\xi.
\end{equation}

On choisit $\phi \in \mathcal{S}(F^n)$ tel que $\phi(0) = 1$. Comme $\widetilde{f}_\pi = f_\pi$ pour tout $\pi \in Temp(PG_{2n})$, d'apr�s le lemme \ref{limitezeta}, on a
\begin{equation}
\begin{split}
\int_{Z_{2n}(H_n \cap N_{2n}) \backslash H_n} W_{\widetilde{f}}(1, \xi) \theta(\xi)^{-1} d\xi &= \lim_{s\rightarrow 0^+} n\gamma(s, 1, \psi) J(s, W_{\widetilde{f}}(1, .), \phi) \\
&= \lim_{s\rightarrow 0^+} n\gamma(s, 1, \psi) \int_{Temp(PG_{2n})}J(s, W_{f, \pi}(1, .), \phi) d\mu_{PG_{2n}}(\pi).
\end{split}
\end{equation}

D'apr�s l'�quation fonctionnelle, on a
\begin{equation}
\begin{split}
&\int_{H_n \cap N_{2n} \backslash H_n} W_f(1, \xi) \theta(\xi)^{-1} d\xi = \\
&\lim_{s\rightarrow 0^+} n\gamma(s, 1, \psi) \int_{Temp(PG_{2n})}J(1-s, \widetilde{W_{f, \pi}(1, .)}, \widehat{\phi}) c(\pi) \gamma(s, \pi, \Lambda^2, \psi)^{-1} d\mu_{PG_{2n}}(\pi).
\end{split}
\end{equation}

Cette derni�re limite est �gale �
\begin{equation}
\int_{Temp(SO(2n+1)/Stab} J(1, \widetilde{W_{f, T(\sigma)}(1,.)}, \widehat{\phi}) c(T(\sigma)) \frac{\gamma^*(0, \sigma, Ad, \psi)}{|S_\sigma|} d\sigma.
\end{equation}

On conclut gr�ce au lemme \ref{zetabeta}.
\end{proof}

\section{Formule de Plancherel}

On note $Y_n = H_n \backslash G_{2n}$. On dispose d'une surjection $f \in \mathcal{S}(G_{2n}) \mapsto \varphi_f \in \mathcal{S}(Y_n, \theta)$ avec
\begin{equation}
\varphi_f(x) = \int_{H_n} f(hx) \theta(h)^{-1} dh,
\end{equation}
pour tous $x \in Y_n$. Pour $\pi \in Temp(G_{2n})$ et $f_1,f_2 \in \mathcal{S}(G_{2n})$, on pose
\begin{equation}
(f_1,f_2)_{Y_n, \pi} = \sum_{W \in \mathcal{B}(\pi, \psi)} \beta(R(f_1)W)\overline{\beta(R(f_2)W)},
\end{equation}
o� $\mathcal{B}(\pi, \psi)$ est une base orthonorm�e de $\mathcal{W}(\pi, \psi)$.

\begin{theoreme}
Soit $\varphi_1, \varphi_2 \in \mathcal{Y_n}$, il existe $f_1, f_2 \in \mathcal{S}(G_{2n})$ tel que $\phi_i = \phi_{f_i}$ pour $i = 1,2$. On a
\begin{equation}
(\phi_1, \phi_2)_{L^2(Y_n)} = \int_{H_n} f(h) \theta(h)^{-1} dh,
\end{equation}
o� $f = f_1 * f_2^{*}$, on note $f_2^*(g) = \overline{f_2(g^{-1})}$. On pose alors
\begin{equation}
(\varphi_1, \varphi_2)_{Y_n, \pi} = \int_{H^P_n \cap N_{2n} \backslash H^P_n} \beta(W_{f,\pi}(\xi_p,.) \theta(\xi_p)^{-1} d\xi_p,
\end{equation}
pour tous $\pi \in Temp(G_{2n})$. Alors on a
\begin{equation}
(\phi_1, \phi_2)_{L^2(Y_n)} = \int_{Temp(SO(2n+1))/Stab} (\varphi_1, \varphi_2)_{Y_n, T(\sigma)} \frac{|\gamma^*(0, \sigma, Ad, \psi)|}{|S_\sigma|}d\sigma.
\end{equation}
\end{theoreme}

\begin{proof}
\end{proof}





\end{document}






